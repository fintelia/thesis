%\section{Conclusion}
\label{s:concl}

This thesis assesses the evolution of performance penalties for
mitigations against transient execution side-channel attacks by
measuring their overheads across several generation of Intel and AMD
CPUs, and introduces the \sys kernel design which eliminates as much
as half this overhead on OS heavy workloads.

Although the time passed since the first mitigations is short,
we observe a few trends: 1) on an OS-intensive workload the
overhead of mitigations has declined from over 30\% on our oldest
processor to under 3\% on the most recent CPUs; and 2) JavaScript
sandboxing stresses different mitigations, whose overheads are
significant and unchanged across successive generations.

A further analysis of individual mitigations shows that the hardware
improvements in successive generations have reduced the overhead of
some mitigations to something small, while the overheads for Spectre
V1 and V2 mitigations haven't changed across processor generations and
are significant.  We speculate, however, that it should be possible to
reduce these overheads of these mitigations with hardware changes too;
for example, the Spectre V1 mitigation has a recognizable pattern of a
conditional move followed by a load instruction, which could be
detected by hardware to trigger special handling.

%The \contract
%allows hardware to speculate on many values (but not the values of
%page table entries) and provides software with a mechanism to prevent
%leaking secrets through micro-architectural state.  
The \sys design
shows how \contract can be used to reduce the performance costs of
mitigations on system calls using per-process Q domains and global K
domains.  \sys transparently switches from Q- to K-domain through page
faults, uses temporary mappings to access unmapped physical pages, and
splits data structures into public and private parts.  An evaluation
shows that \sys can run the microbenchmarks of \bench with small
performance overhead compared to a kernel without mitigations: for
18 out of 30 \bench microbenchmarks, \sys's performance is within 5\%
of the performance without mitigations.
On Broadwell, \sys overall has a geometric mean slowdown of 16\% on \bench compared to 32\% for Linux.
% Although \sys is research kernel, we are hopeful that its
% ideas can carry over to production monolithic kernels.
Although \sys is research kernel, we are hopeful that the ideas of this thesis can drive further progress in production kernels and web browsers.
