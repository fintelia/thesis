\section{Design}
\label{s:design}

Under the assumption of the unmapped speculation contract, this
section describes how \sys can reduce the cost of mitigations for
system calls.  \autoref{ss:overview} provides an overview of \sys's
design with subsequent sections providing more detail about \sys's
switch between protection domains (\autoref{ss:switch}), about the
mitigations used by \sys when mitigations are necessary
(\autoref{ss:mitigations}), \sys's kernel text (\autoref{ss:ktext}),
\sys's memory management modifications (\autoref{ss:mm}), \sys's
process management split (\autoref{ss:proc}), and \sys's file system
split (\autoref{ss:fs}).

\subsection{Overview}
\label{ss:overview}

\begin{figure}[t]
  \begin{center}
    \begin{tikzpicture}[>=latex]

  \tikzstyle{A}=[thick,rectangle, draw, minimum width=1cm,minimum
    height=1.5cm, align=center];
  \tikzstyle{Q1}=[thick,rectangle, draw, minimum width=1cm,minimum
    height=1.5cm, align=center,fill=green!30];
  \tikzstyle{Q2}=[thick,rectangle, draw, minimum width=1cm,minimum
    height=1.5cm, align=center,fill=blue!30];
  \tikzstyle{K}=[thick,rectangle, draw, minimum width=1cm,minimum
    height=1.5cm, align=center,fill=red!30];
  \tikzstyle{BIG}=[minimum height=3cm];
  
  \node [Q1,fill=gray!30] (q1) {};
  \node [Q1,xshift=1.2cm,fill=gray!30] (qk1) {};
  \node [Q2,xshift=2.4cm,fill=gray!30] (qk2) {};
  \node [Q2,xshift=3.6cm,fill=gray!30] (q2) {};
%  \node [K,xshift=4.8cm,fill=gray!30] (k) {};

  \node [left=of q1.west,align=left,xshift=.5cm] (u) {Public};

  \node [Q1,BIG,below=of q1.south,yshift=1cm,fill=white] (q11) {};
  \node [K,BIG,below=of qk1.south,yshift=1cm] (qk11) {};
  \node [Q2,BIG,below=of q2.south,yshift=1cm,fill=white] (q21) {};
  \node [K,BIG,below=of qk2.south,yshift=1cm] (qk21) {};
  
  \node [Q1,below=of q1.south,yshift=1cm,minimum height=0.25cm] (q11a) {};
  \node [K,below=of qk1.south,yshift=1cm,minimum height=0.25cm,fill=green!30] (qk1a) {};
  \node [K,below=of qk2.south,yshift=1cm,minimum height=0.25cm,fill=green!30] (qk2a) {};

  \node [Q1,below=of q1.south,yshift=0cm,minimum height=0.25cm,fill=yellow!70] (q21b) {};
  \node [K,below=of qk1.south,yshift=0cm,minimum height=0.25cm,fill=yellow!70] (qk1c) {};
  \node [K,below=of qk2.south,yshift=0cm,minimum height=0.25cm,fill=yellow!70] (qk2c) {};
  \node [Q2,below=of q2.south,yshift=0cm,minimum height=0.25cm,fill=yellow!70] (q21b) {};

  \node [Q1,below=of q1.south,yshift=-0.5cm,minimum height=0.25cm] (q11s1) {\small Stack};
  \node [K,below=of qk1.south,yshift=-0.5cm,minimum height=0.25cm] (qk1s1) {\small Stack};
  \node [K,below=of qk2.south,yshift=-0.5cm,minimum height=0.25cm] (qk2s1) {\small Stack};

  \node [K,below=of qk1.south,yshift=-1.25cm,minimum height=0.25cm] (qk2s2) {\small Stack};
  \node [K,below=of qk2.south,yshift=-1.25cm,minimum height=0.25cm] (qk2s2) {\small Stack};
  \node [Q2,below=of q2.south,yshift=-1.25cm,minimum height=0.25cm] (q21s2) {\small Stack};
  
%  \node [K,below=of k.south,yshift=1cm,minimum height=0.25cm,fill=green!30] (k1a) {};
%  \node [K,below=of k1a.south,yshift=1cm,minimum height=1.0cm] (k1b) {};
%  \node [K,below=of k1b.south,yshift=1cm,minimum height=0.25cm,fill=blue!30] (k1c) {};

  \node [left=of q11.west,align=left,xshift=.5cm] (u) {Private};

%  \node [Q1,below=of q11b.south,yshift=0cm] (q12) {};
%  \node [K,below=of qk1c.south,yshift=0cm] (qk12) {};
%  \node [Q2,below=of q21b.south,yshift=0cm] (q22) {};
%  \node [K,below=of qk2c.south,yshift=0cm] (qk22) {};
  
%  \node [K,below=of k1c.south,yshift=1cm] (k2) {};

%  \node [left=of q12.west,align=left] (u) {Stack};
  
  \node [Q1,below=of q11.south,yshift=1cm,fill=gray!30] (q13) {\small No \\ \small miti- \\ \small gat- \\ \small ions};
  \node [K,below=of qk11.south,yshift=1cm] (qk13) {\small With \\ \small miti- \\ \small gat- \\ \small ions};
  
  \node [Q2,below=of q21.south,yshift=1cm,fill=gray!30] (q23) {\small No \\ \small miti- \\ \small gat- \\ \small ions};
  \node [K,below=of qk21.south,yshift=1cm] (qk23) {\small With \\ \small miti- \\ \small gat- \\ \small ions};
  
%  \node [K,below=of k2.south,yshift=1cm] (k3) {};

  \node [left=of q13.west,align=left,xshift=.5cm] (u) {Text};

  %% user space
  
  \node [A,above=of q1.north,yshift=-1cm,fill=green!30] (a1) {};
  \node [A,above=of q2.north,yshift=-1cm,fill=blue!30] (a2) {};

  \node [A,above=of qk1.north,yshift=-1cm,fill=green!30] (a11) {};
  \node [A,above=of qk2.north,yshift=-1cm,fill=blue!30] (a21) {};

  \draw[thick,color=red] (q1.north) ++(-2cm,0) -- ++(7.5cm,0)  node {};

  \node [right=of a1.east,xshift=3cm,align=left] (u) {User\\space};
  \node [right=of q1.east,xshift=3cm,align=left] (k) {Kernel\\space};

  \node [below=of q13.south,yshift=1cm,align=left] {Q1};
  \node [below=of qk13.south,yshift=1cm,align=left] {K1};
  
  \node [below=of q23.south,yshift=1cm,align=left] {Q2};
  \node [below=of qk23.south,yshift=1cm,align=left] {K2};
  
%  \node [below=of k3.south,yshift=1cm,align=left] {K};
  
\end{tikzpicture}


  \end{center}
  \vspace{-2\baselineskip}
  \caption{Overview of \sys's address space layout with two processes
    (indicated by the colors green and purple). Each process has a Q
    and K domain. Q domains have access to public data (the grey
    color) and per-process kernel data; the
    white private region is unmapped kernel data. Each domain also has
    its own stack and kernel text. In the Q domain, the kernel text
    has no mitigations. The K domains map all memory, including
    sensitive memory (indicated by red); all K domains have the same
    memory layout. Data structures that are shared across processes,
    such as pipes or file pages, can be mapped in multiple Q domains,
    as indicated by the yellow color.}
\label{fig:overview}
\end{figure}

\sys's design maintains two page tables per process.  One page table
defines a process-specific view of kernel memory.  When a process is
running with that page table, we say it is running in its
\textit{quasi-visible} domain (or Q domain for short), and with its Q page
table.  Following the unmapped speculation contract, \sys assumes any
kernel memory mapped by the Q page table can be leaked to the
currently running process.  Instead of using mitigations to prevent
leaks of kernel memory, \sys arranges for the mappings in the Q page
table to be such that they contain no sensitive data of other
processes.

When the process needs to access data that is not mapped in the Q page
table, it can switch to its other page table, which maps all physical
memory, including memory that contains sensitive data.  When a process
is running with this page table, we say the process is executing in
its \textit{K} domain with its K page table.  In its K domain, the process
runs with the same mitigations as Linux currently uses.

This design allows many system calls to execute in the Q domain,
with no mitigation overhead.  As a simple example,
\texttt{getpid} does not access any sensitive data; it needs
access to only the kernel text and its own process structure.  A more
interesting example is mapping anonymous memory: this requires access
to the process's own page table and to the memory allocator, but not
other processes' page tables or pages.

\autoref{fig:overview} shows the address space layout in \sys in more
detail.  Each process has a Q and K view of memory.  When a process is
running in user space it runs in its Q domain (with no secrets mapped
in the Q page table).  When a user process makes a system call it
enters the kernel but stays in its Q domain.  The Q domain maps public
kernel memory, Q-visible kernel memory, the process's Q domain stack,
and the kernel text without mitigations.

If a system call needs access to memory in the K domain, \sys performs a
switch from its Q domain to its K domain. We refer to the switch from a Q
domain to a K domain as a \textit{world switch}, because kernel code in
a Q domain runs without most mitigations and the kernel code in the K domain
runs with full mitigations. Furthermore, the process switches from its
Q domain stack to its K domain stack.  The K domain, with access to
all kernel memory, can then execute the rest of the system call with
full mitigations.

Achieving good performance in \sys depends on avoiding world switches.
To reduce the number of world switches, \sys maps kernel data
structures that contain no sensitive data into every Q domain.  For
example, all Q domains map x86 configuration tables (IDT, GDT), some
memory allocator state, etc.  On the other hand, kernel data structures
that contain application data, such as process memory or saved register
state, are not mapped into Q domains unless that process should have
access to that data.

\subsection{World switch}
\label{ss:switch}

One of the challenges in \sys's design is that a system call often does
not know upfront whether it will need to execute in the Q domain or in
the K domain.  For example, a \texttt{read} system call might be able to
execute purely in the Q domain, or might need access to sensitive data
from the K domain, depending on the file descriptor that the process is
reading from, and depending on whether this Q domain already has some
sensitive data mapped or not.  To support this, \sys's design allows
a system call to start executing in the Q domain, and switch to the K
domain later as needed.

\sys allows the Q domain to trigger a world switch either
\textit{intentionally} or \textit{transparently}.  If the code
determines that it must switch to the K domain, it can intentionally
invoke the function, \texttt{kswitch()}, to perform a world switch.
When \texttt{kswitch()} returns, the kernel thread is now executing in
the K domain, and has access to all memory.  If the Q domain needs
access to specific sensitive data which might or might not be already
mapped, the Q domain can attempt to access the virtual address of that
data.  If the data is already mapped in the Q domain, the access will
succeed, no world switch happens, and the Q domain can continue
executing.  If the data is not mapped, the Q domain triggers a page
fault, which transparently triggers a world switch.  Once the page
fault returns, the kernel thread is now executing in the K domain, as
if it called \texttt{kswitch()}.  Compared to making an intentional
call to \texttt{kswitch()}, the transparent approach incurs a slight
overhead for executing the page fault, but allows large sections of
the kernel to be kept completely unmodified, and allows the Q domain
to elide a world switch altogether if the data is already mapped in
the Q domain.

The above design requires that a kernel thread can start executing in
the Q domain and transparently switch to executing in the K domain.
This means that any addresses that the kernel thread is referencing,
including pointers to data structures, stack addresses, and function
pointers, remain the same.  To achieve this, \sys ensures that the
layout of the Q domain and the K domain match.  In particular, all
data structures in the Q domain must appear at the same address in the
K domain, and the kernel code (text) is located at the same address
(even though the code is slightly different, as described in \autoref{ss:ktext}).

The stack requires particular care because a kernel thread that is
processing sensitive data in the K domain could inadvertently write
that data to the stack.  For example, a \texttt{read()} system call
from \texttt{/dev/random} needs to switch to the K domain to access the
system-wide randomness pool.  However, the pseudo-random generator code
might spill some of its state to the stack, depending on the compiler's
choices.  If the stack is accessible from the Q domain, the sensitive
data could in turn be leaked during the next entry into the Q domain by any
thread within the same process.  At the same time, if the
K domain stack was separate from the Q domain stack, pointers to stack
locations before a world switch would no longer work after a world switch.
To reconcile these constraints without having to rely on any dedicated
compiler support, \sys maps a distinct kernel stack for each domain at the
virtual address range and copies the Q domain stack contents to the K domain
stack during a world switch.


\subsection{Mitigations}
\label{ss:mitigations}

% \fk{this section needs to be fleshed out and needs careful review}

\begin{figure}
\centering
\small
\begin{tabular}{lccc}
{\bf Transient execution vulnerability} & {\bf User/Q domain} & {\bf K domain} & {\bf Context Switch} \\

\midrule
L1TF & x & x & \\

\midrule
Spectre V1 & & x & \\
Bounds Check Bypass Store & & x &\\
Meltdown && x &\\
Speculative Store Bypass &	 & x &  \\

\midrule
Spectre V2 & & x & x \\
Microarchitectural Data Sampling && x & x\\
\qquad (Fallout, RIDL, Zombie Load, etc.) \\

\midrule

LazyFP &&& x \\
SpectreRSB &&& x\\

\midrule
PortSmash & \multicolumn{3}{c}{Not applicable} \\
Load Value Injection & \multicolumn{3}{c}{Not applicable} \\
Meltdown-PK (protection key bypass) & \multicolumn{3}{c}{Not applicable} \\
Meltdown-BR (bounds check instr. bypass) & \multicolumn{3}{c}{Not applicable} \\
Read-only Protection Bypass & \multicolumn{3}{c}{Not applicable} \\

\end{tabular}
\caption{The mitigations implemented in software by \sys{}.}
\label{fig:mitigations}
\end{figure}

\autoref{fig:mitigations} shows the known transient execution
attacks~\cite{hill:survey,sok:transient}, organized by the mitigations
needed to address those attacks in \sys's design.  The columns indicate where the mitigations are needed: while
executing in user-mode or the Q domain; while executing in the K domain;
and when context-switching between processes.

% The first attack, System Register Read, requires mitigation in user-space
% to prevent adversarial processes from obtaining the contents of sensitive
% system registers.  However, no mitigation is required when running in
% the K domain, since we do not expect the K domain to issue speculative
% system register reads (much as in Linux).

The L1TF attack allows leaking the contents of the L1 cache if
there are partially-filled-in entries in the page table.  We think of
this attack as a violation of the USC (see \autoref{fig:attack-by-state}),
but a simple microcode fix, as well as clearing unused page table entries,
makes the system agree with the USC, and avoids the L1TF attack.  Since
L1TF allows leaking the contents of any data, \sys applies the mitigations
both in user-space, Q domain, and K domain.

The next category of attacks requires no mitigations in either user-space
or Q domain.  Specifically, Spectre variants that bypass bounds checks
require mitigation in the K domain, since there is sensitive memory
contents that could be leaked as a result of a speculative check bypass.
However, there is no sensitive data that can be leaked in the Q domain,
owing to \contract{}.  Similarly, no mitigations are required on a
context switch, since these attacks can only leak data from the current
protection domain.

Meltdown also falls in this category, but for a different reason.
Meltdown allows an adversary to bypass the user-kernel boundary check
in the page table.  \sys's use of a separate page table for the Q and
K domains ensures that Meltdown cannot leak any confidential data, since
no confidential data is available in the Q domain.  Recent fixes from Intel resolve the Meltdown attack in a way that avoids the need for
software mitigations.

The next category of attacks require mitigation both in the K domain and
on context switch. Both Spectre V2 and MDS can allow an adversary to
obtain sensitive data either from the OS kernel or from another process.
However, no mitigations for these attacks are needed in the Q domain
due to \contract{}: there is no sensitive data to leak in the Q domain
of the currently running process.

For some attacks, such as LazyFP and SpectreRSB, mitigations are only
required on context switch, because the attacks involve process-to-process
leakage.

Finally, a number of attacks are not applicable to \sys's simpler design,
in contrast to Linux.  For example, \sys does not support SGX, does
not support running virtual machines, and does not use certain hardware
features (such as hardware bounds-check instructions or protection keys).


\subsection{Kernel text}
\label{ss:ktext}

Some of the mitigations involve changes to the executable kernel code
(text), such as the use of retpolines in place of indirect jumps.
These mitigations impose a performance cost, but they are not needed
when executing in the Q domain.

A na\"ive approach might be to compile the kernel code twice, with
different compiler flags for mitigations, and load the two different kernel
binaries in the Q and K domains respectively. However, this would break \sys's
page fault triggered world switches because after completing the switch,
execution would resume with the same instruction pointer and stack contents
from before the switch but neither would be meaningful in the new text
segment.

Instead we need the two version to have matching instruction addresses and stack
layouts. \sys achieves this by compiling the kernel only once, but then making
two copies of the code at runtime. One copy is mapped into all the K domains,
and the other into all the Q domains \textit{but at the same virtual address
as in the K domains}. Switching between the two is seamless.

At boot time, in a process inspired by Linux's \textsc{alternative}
macro~\cite{lwn:alternative}, \sys locates each \texttt{call} or
\texttt{jmp} in the Q text segment pointing to a retpoline thunk, and
replaces them with the instruction that retpoline emulates. One
complication is that indirect call instructions are only 2 or 3 bytes,
compared to the 5 that a direct call instruction takes. If we tried
to pad with a \texttt{NOP} instruction before or after, the pair would not execute
atomically, so instead we prepend indirect calls with several
repetitions of the CS-segment-override prefix, which is always ignored
in 64-bit mode.


\subsection{Memory management}
\label{ss:mm}

Memory allocation in \sys is complicated by the fact that the contents
of free pages may contain sensitive data.  In particular, if a page was
freed by one process, its contents must be erased before the page can
be mapped in another Q domain.  Zeroing out pages on every allocation
would be costly, in particular when allocating kernel data structures,
which do not otherwise require the memory to be zero-filled.

To avoid the overhead of repeatedly zeroing kernel pages, \sys implements
a sharded allocator for kernel memory.  Each Q domain has its own pool
of pages for allocation, and the K domain keeps all of the kernel memory
that is not part of any Q domain.  \sys transfers memory between these
shards in batches to amortize the world switch overhead.  Keeping a pool
of kernel memory in a Q domain allows the kernel to repeatedly allocate
and free memory within a Q domain with little overhead.

The other category of memory managed specially by \sys is public memory.
\sys maintains a single pool of public pages, with separate functions,
\texttt{palloc()} and \texttt{pfree()}, for allocating and freeing in
that pool.  All public-pool pages are mapped in every Q domain.


\subsection{Process management}
\label{ss:proc}

When the \sys kernel switches from executing one process to another, it
must perform a world switch, to ensure that confidential data does not
leak across processes (such as the saved CPU registers that the kernel
might keep on the stack).  However, if a multi-threaded application
is running, there is no security reason to perform a world switch when
switching between multiple threads in the same process---all of the
threads have the same privileges and have access to the same process
address space.

To avoid mitigation overhead when switching between threads in the
same process, \sys splits the process descriptor, \texttt{struct
  proc}, into two parts.  The first part stores sensitive process
state, such as the saved CPU registers, and is not public.  The second
part stores metadata about the process, such as the PID, the run
queue, the scheduler state, etc.  This part is public and is used by
the scheduler when deciding what thread to execute next.  As a result,
the scheduler can pick the next thread without incurring a world
switch.  Furthermore, if the next thread happens to be from the same
process, the context switch code can also avoid performing a world
switch.  Existing scheduler policies that favor picking threads from
the same process mesh well with this approach.

\subsection{File system}
\label{ss:fs}

File system workloads involve access to several kernel data structures,
including the inode cache and the page cache (containing file data).
Inodes are challenging for \sys to deal with because they are smaller than
a page, so it is not feasible to map them individually into a Q domain.
However, achieving good performance for file system operations requires
being able to access an inode without a world switch.  To reconcile
this conflict, we chose to make all inode structures public in \sys,
similar to our approach for splitting the \texttt{proc} structure above.
If the inode had sensitive data (such as extended attributes), that
part of the inode structure would need to be split off into a separate
private structure, along the lines of how we split off the part of the
\texttt{proc} structure storing saved CPU registers.

File data pages are not public, because their contents might be sensitive.
\sys implements an optimization that allows it to access file
contents without a world switch.  In particular, after \sys checks
the permissions on a file, it reads or writes the contents of a file
page by temporarily mapping the corresponding physical page of memory
into its Q domain's address space.  This allows the Q domain to access
that specific memory page without the risk of leaking other pages;
as a result, no mitigations or world switches are needed.  When the Q
domain is finished with the file read or write, it unmaps the page and
issues a TLB shootdown, in case the file is later truncated and the page
gets reused for other data.


\subsection{Pipes}

Pipes are different from many of the other kernel data structures discussed so
far in that their contents shouldn't be visible globally, but their state can be
associated with multiple processes at a time. \sys's goal is to ensure that if a
reader and writer of a pipe run on different cores, then they don't incur world
switches when they access the pipe. To achieve this, we store a pipe's data
structures in shared memory regions between Q domains. These shared regions are
lazily mapped into Q domains the first time a process accesses a pipe (doing the
mapping on \texttt{fork} would cause unnecessary overhead), and unmapped when
the last reference to the pipe within a Q domain is closed. 

When a pipe becomes full or empty, the caller blocks on a condition
variable. Subsequent reads or writes can observe which processes are blocked and
add them to the scheduler run queue if appropriate. Neither of these operations
requires access to any secret data so no world switch is triggered until a new
process is scheduled. Thus, if the core remains idle until the blocking thread
is added back to the run queue, the cost of a world switch is avoided.

\subsection{Discussion}

\sys's design assumes that there are no secrets in the Q domain that
need to be hidden from the user-level process.  For many secrets, they
can be protected by placing them in the K domain, such as the seed
of a system-wide randomness generator.  However, address-space layout
randomization (ASLR) for the kernel address space is difficult to protect
in this fashion, because kernel addresses must be used in the Q domain,
and the addresses must match up between the Q domain and the K domain
in order for world switches to work.  (Note that the initial seed that
is used to randomize layout could be protected in the K domain, but the
resulting randomized layout cannot be protected.)  As a result, kernel
ASLR in \sys is susceptible to leakage of addresses through transient
execution side-channels.

Our \sys prototype does not include an optimized in-kernel network stack, but a
reasonable approach might be to treat all network data as public, leaving
it up to the application to encrypt any sensitive information sent over
the network.  This meshes well with the recent trends in widespread use
of TLS for network security, and allows for network operations to achieve
high performance in \sys because no mitigations or world switches are
required, and all network processing can stay in the Q domain.

Hyperthreading is a source of many possible transient execution leaks,
because a significant amount of microarchitectural state is shared
between the execution contexts.  However, many Linux systems continue
to run with hyperthreading \emph{enabled}, despite these risks, because
of the high performance overhead they would incur if hyperthreading was
entirely disabled.  \sys does the same.
