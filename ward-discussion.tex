\section{Discussion}
\label{s:ward-discussion}

\paragraph{Future vulnerabilities.}

It is likely that there are further transient
execution attacks either under embargo or yet to be discovered. Based on trends
in the existing attacks, we believe that \sys should be well positioned to
address them: so far, mitigations developed for Linux have been suitable to
directly copy into \sys. Since many need to run only at K domain entry/exit
instead of every user-kernel boundary crossing, the same defenses in \sys
might be cheaper to apply than they would be for Linux.

\paragraph{Linux.}

%% The results in \autoref{s:eval} shows that Ward's design can avoid
%% incurring the mitigation overhead for important system calls by executing
%% them in the Q domain without mitigation.  Although the evaluation involves
%% a research kernel with less features than a production kernel like
%% Linux,

We are optimistic that Ward's techniques could also benefit monolithic
production kernels for two reasons.
First, \sys and Linux are in the same ballpark in terms of system call
performance on \bench.  Out of the 30 microbenchmarks, \sys is faster than
Linux on 18 of them, and slower on 12.
%% Although the kernel designs are different in some
%% respects (and \sys is quite a bit simpler than Linux), this suggests
%% that the relative costs of mitigations on \sys are a reasonable proxy
%% for that on Linux.
Second, as shown in \autoref{fig:linuxslowdown} (\autoref{s:motivation})
Linux incurs a significant overhead for mitigations on \bench and that
overhead is in line with the overhead that \sys's Linux-style mitigations
incur on \bench (see
\autoref{fig:slowdown}).  Some systems calls experience more overhead in
\sys, because they implement less functionality (e.g., \texttt{getpid}),
but the corresponding calls in Linux also incur significant overhead.
Some systems calls in \sys have less overhead than Linux, because they are
not as efficient; for example, big and huge \texttt{mmap} in \sys requires
an update of its radix-tree VM data structures~\cite{clements:radixvm},
while Linux just inserts the new region into a list. Linux may see a
bigger payoff for those system calls with \sys's design than \sys.

A question is how much effort is required to incorporate
\sys's techniques into a production kernel such as Linux.
%% To shed some
%% light on this question, we have started to modify the Linux kernel
%% to incorporate \sys's ideas.
Our preliminary efforts have proven
encouraging: we found that we could leverage existing infrastructure
for KPTI to maintain Q domain and K domain page tables.  We implemented
a \texttt{switch\_world} function in Linux, which switches to the K
domain and copies the Q stack to the K stack.  We modified the Linux
page-fault handler to call this function when it encounters a page
fault while running with the Q page table. This allows the Linux kernel
to run as normally with a transparent world switch on each system call.
We refactored the \texttt{struct task\_struct} into a Q-private and secret
part, allowing the \texttt{gettid} system call to run entirely in the
Q domain.  This gives us some indication that the basic approach of \sys
could be made to work in Linux, although an open question is how to best
re-design Linux's data structures to fit \sys's design.
