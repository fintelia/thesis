\section{Discussion}
\label{s:discussion}

We present some thoughts for computer architects and
operating systems developers that can address the lingering slowdowns from transient execution attacks.

\subsection{Hardware mitigations for Spectre V1}

One takeaway from the previous sections is the continued impact of Spectre V1.
There are no hardware mitigations available for the attack in high performance commercial CPUs.
And yet, despite being among the first transient execution attacks discovered, it still presents a significant---and largely unchanging---overhead when mitigated in software.

Because Spectre V1 mitigations are specifically applied by JIT engines doing code generation, they also may present a unique opportunity for computer architects.
The JIT annotates each vulnerable gadget with a leading \texttt{cmov} instruction.
This pattern of a conditional move followed by a load instruction could be detected by hardware to trigger special handling.

Even if this approach proves unworkable, that doesn't rule out hardware acceleration for Spectre V1 mitigations.
JIT engines generate code on the fly based on the processor they are running on, which means that unlike native applications, the author of a given JavaScript application doesn't need to be involved in porting/recompiling it to leverage a new ISA extension.
And since current web browsers generally receive new updates on a six
week release cycle, any new hardware could be leveraged quickly.

\subsection{Speculative Store Bypass}

Speculative Store Bypass Disable was initially implemented in microcode, and while we cannot tell whether more recent CPUs include actual hardware changes as well, the performance overhead hasn't improved.
This attack in particular also emphasizes the need to look at the performance impacts of transient execution attacks across representative workloads.
Despite being disabled by default, Speculative Store Bypass Disable incurs a substantial overhead on JavaScript execution in web browsers---one of the most common workloads run by end-users.

Intel's inclusion of a hardware capability to detect whether a
processor is vulnerable to Speculative Store Bypass (without a way to toggle it) strongly
suggests that they believe future hardware will be able to prevent
the attack with negligible overhead.

\subsection{Hypertheading safety}

The discovery of more and more transient execution attacks has eroded trust that CPUs are capable of fully preventing information leaks between concurrently running tasks scheduled on hyperthread pairs.
This is particularly worrisome given that both many of the existing transient execution attacks work in this setting, and that there are a substantial number of other microarchitectural components that are shared between hyperthreads, which could also present side channels.

Linux by default considers this attack vector not worth worrying about, but a heavy-handed approach that's been adopted by certain cloud providers has been to entirely disable hyperthreading when running client virtual machines.
This approach comes at a high cost because the CPU is no longer able to benefit
from the increased utilization that hyperthreading allows~\cite{intelht}.

There is however another option that operating system developers could pursue.
Since it is safe to simultaneously run two mutually trusting processes
on hyperthread pairs, a more sophisticated scheduling policy could
ensure that only processes that are safe to collocate
are assigned to adjacent logical processors.

