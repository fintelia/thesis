\section*{Goal and threat model}

\sys's goal is to reduce the performance cost of mitigations for transient
execution attacks.  In principle, \sys's techniques can reduce not only
the cost of software mitigations, but also allow processor designers
to avoid costly mitigations in hardware.  Practically, however, it is
difficult for us to disable hardware mitigations in the newest processors.
%% they appear to have significant performance costs, as shown in the
%% previous section.
Therefore, this paper focuses on reducing the overhead
of software mitigations, and experimentally measures their effect on the
previous generation of CPUs, where we can avoid mitigations altogether.
We hope that \sys's design can allow processor designers to regain some
of the absolute performance lost due to hardware mitigation costs.

Our threat model targets scenarios where the adversary
and the victim are both running code on the same computer.  This might
arise either in a server setting, where both are running on a cloud
computing platform, or in a client device, where the adversary code is
a malicious application or web site.

Canella et al.~\cite{sok:transient} discuss transient execution attacks
in detail, but the salient points of the attack boil down to four steps.
First, the processor speculatively executes some code, which accesses
sensitive data that the victim wants to keep secret.  Second, during the
speculative execution, the processor updates microarchitectural state
in a way that depends on the sensitive data (e.g., bringing in cache
lines into a shared L3 cache whose addresses depend on the sensitive
data).  Third, the processor aborts the speculative execution, but does
not fully roll back all of its side effects (e.g., changes to the L3
cache), because doing so would be prohibitively expensive in hardware.
Fourth, the adversary observes these side effects (e.g., using timing
measurements), which allows the adversary to infer the sensitive data.

What makes transient execution attacks challenging to mitigate in an
OS kernel is a combination of two factors.  The first is that an
adversary can trigger an OS kernel to speculatively execute code that
leads to leakage of sensitive data.  Even though the adversary cannot
inject their own code to execute in the kernel, the adversary can often
have significant influence on what existing kernel code gets executed in
speculative execution, by specifying particular system call arguments or
setting up micro-architectural CPU state such as the branch predictor.
The second factor is that an OS kernel has access to all of the state
on the computer.  This means that an adversary running in one process
can trick the kernel into leaking state from any other process on the
same computer.

Current OS kernel designs, such as Linux, have two approaches for
mitigating transient execution attacks.  The first approach is to make
sure that the CPU does not speculatively execute any code that could
end up accessing sensitive data.  This approach includes techniques such
as retpolines and other speculation barriers.  The second approach is to
make sure that sensitive data is flushed from microarchitectural state,
such as flushing CPU caches and buffers when returning from a system
call or when context-switching between processes.  Both incur significant
performance overheads.

Transient execution attacks can leak data across many protection domain
boundaries, such as leaking secrets from the kernel to an adversary's
process, or leaking secrets from one process to a different process,
or even leaking secrets within a single process that implements its
own internal protection domains.  Much like in the Linux kernel, the
focus of \sys is on preventing leakage between processes, as well as
preventing leakage from the kernel to a process.  \sys's approach to
preventing cross-process leakage is the same as Linux (flushing state),
but \sys has a novel approach for efficiently preventing kernel-to-process
leakage of memory contents, as we describe in the next section.

Although \sys addresses all known transient execution attacks, the focus
of this paper is on attacks that allow the adversary to leak the contents
of arbitrary memory, which is especially important in an OS kernel.
\sys handles other transient execution attacks, such as leaking the
contents of sensitive data already present in the CPU (e.g., x86 MSRs),
in the same way as Linux does.

Attacks that do not leverage transient execution to leak data are also
out of scope for this paper, since they are orthogonal to the key
challenge of transient execution leakage.  In particular, we do not
consider attacks that leverage physical side channels (such
as Rowhammer or RAMbleed), cache side channels (such as cache timing
attacks), power side channels, etc.


\section*{Approach: Unmapped speculation contract}

\sys's design for mitigating transient execution attacks relies
on page tables.  Specifically, if a page of physical
memory is not referenced by any entry in the current page table or
TLB, speculative execution cannot access any sensitive data stored in
that page, because the page doesn't have a virtual address to access
it by.

% Processors typically do
% not speculate on the physical address bits in page table entries, and as a result, the
% lack of a page table entry serves as a barrier to speculative execution.

A contribution of this paper lies in articulating a hardware/software
contract---which we call the \emph{unmapped speculation contract}---that
captures the above intuition.  The contract aims to provide a strong
foundation for keeping data confidential, which is typically stated as
non-interference.  Non-interference can be thought of by considering
two system states, $s$ and $s'$, that differ only in sensitive data,
which should not be observable by an adversary.  A system ensures
non-interference if an adversary cannot observe any differences in how
the system executes starting from either $s$ or $s'$.

\paragraph{Single-core USC.} To formally state the unmapped speculation contract, we start with a
single-core definition.  We use $A(\cdot)$ to refer to the state of
the CPU, including all architectural and micro-architectural state,
but excluding the contents of memory, and we use $M(\cdot)$ to refer
to the contents of \emph{mapped} memory, i.e., the contents of every
valid virtual address based on the committed page table in that state.
We define the contract by considering a single clock cycle of the
processor's execution, $\textrm{step}(\cdot)$, which includes any
speculative execution done by the processor on that cycle, and require
that unmapped pages cannot influence it:

\begin{small}
\begin{quote}
$\forall s, s', \\
  \textrm{if~} A(s) = A(s')
  \textrm{~and~} M(s) = M(s'), \\
  \textrm{then~with~} S := \textrm{step}(s)
  \textrm{~and~} S' := \textrm{step}(s'), \\
  \textrm{it~must~be~that~} A(S) = A(S')$
\end{quote}
\end{small}

In plain English, the definition considers a pair of starting states $s$
and $s'$ that should look the same, as far as speculative execution is
concerned, because they have the same CPU state and the same contents
of mapped pages.  They might, however, differ in the contents of some
unmapped physical pages, which contain sensitive data that we would
like to avoid leaking.  The definition then considers the state of
the CPU at the next clock cycle ($S := \textrm{step}(s)$ and $S' :=
\textrm{step}(s')$ respectively), and requires that the CPU architectural
and micro-architectural state $A(\cdot)$, which the adversary might
observe, continues to be the same in those two states.  As a result,
the microarchitectural state could not have been influenced by any
sensitive data not present in $M(s)$.

If the OS kernel does not change the mapped memory in that clock cycle,
$M(\cdot)$ remains the same, and the contract will continue to hold on
the next cycle too.  However, if the OS kernel changes the mapped memory,
the contract allows speculative execution from that point on to use the
newly mapped memory, and the kernel will need to use other mitigations
to defend against transient execution leaks from the newly mapped memory,
if necessary.

The contract specifies how the micro-architectural state, $A(\cdot)$,
can evolve, but does not say anything about how $M(\cdot)$ can change.
This is because the focus of the contract is on transient execution,
which cannot affect the committed architectural state of the system; the
contents of memory is described by the ISA, since it is architectural
state.  In other words, changing the memory requires committing the
execution of some instruction, at which point this is no longer a
transient execution.

\paragraph{Multi-core USC.} In a multi-core setting, the CPU state can be thought of as consisting
of per-core state (e.g., registers, execution pipeline, and root
page table pointer), which we denote with $A_i(\cdot)$ for core $i$,
and the uncore state (e.g., the hardware random number
generator~\cite{ragab:crosstalk}), which we denote with $U(\cdot)$,
shared by all cores.  Similarly, since each core has its own page table,
we index the mapped memory by the core $i$ whose page tables we are
considering, $M_i(\cdot)$.  Finally, we consider the multi-core system
executing a clock cycle on one core at a time, $\textrm{step}_i(\cdot)$.
We assume that $\textrm{step}_i(\cdot)$ does not change $A_j(\cdot)$ for
any $i\neq j$.
With this notation, the multi-core contract says:

\begin{small}
\begin{quote}
$\forall s, s', i, \\
  \textrm{if~} A_i(s) = A_i(s');
               U(s) = U(s');
  \textrm{~and~} M_i(s) = M_i(s'), \\
  \textrm{then~with~} S := \textrm{step}_i(s)
  \textrm{~and~} S' := \textrm{step}_i(s'), \\
  \textrm{it~must~be~that~} A_i(S) = A_i(S')
  \textrm{~and~} U(S) = U(S')$
\end{quote}
\end{small}

This means that speculative execution on core $i$ is allowed to depend on
the state of core $i$, the uncore state, and the memory mapped by core
$i$.  This multi-core formulation allows transient execution to affect
both the core state $A_i(\cdot)$ as well as the uncore state $U(\cdot)$,
at the micro-architectural level.  However, transient execution cannot
affect either of these states in a way that depends on unmapped memory.

Although hardware threads appear to provide separate execution
contexts, with a separate page table for each hardware thread, they
have extensive sharing of core resources.  To capture that, we consider
$A_i(\cdot)$ to include the state of all hardware threads on core $i$,
$\textrm{step}_i(\cdot)$ to include the execution of any hardware thread
on core $i$, and $M_i(\cdot)$ to be the union of memory mapped by all
of the hardware threads on core $i$ (i.e., the union of the page tables
of the threads).  With this model, the contract allows leakage of mapped
memory across hardware threads.


\paragraph{Benefits of the USC.}

The contract helps reconcile security and performance of speculative
execution.  On the one hand, hardware can keep the high performance
provided by out-of-order execution, because the contract allows almost
all forms of speculative execution, as long as data during speculative
execution is accessed through non-speculative TLB entries.  On the
other hand, software can precisely specify what data can and cannot
be used for speculative execution, by configuring page
tables.  For example, if the mapped pages never contain sensitive data,
then no mitigations are needed to defend against transient execution
vulnerabilities.  Finally,
because OS developers expect page faults and TLB misses to be quite
expensive (compared to memory references), \contract doesn't change
their performance expectations: developers already have adapted their
designs to avoid excessive page faults or TLB invalidations.

Although the contract is aspirational, one appealing property of the
contract is that modern computer architectures already effectively aim
to provide such a guarantee.  AMD explicitly states in bold font that
their ``processors are designed to not speculate into memory that is
not valid in the current virtual address memory range defined by the
software defined page tables''~\cite[pg. 2]{amd:speculation}.  Intel
has no explicit position about this contract, but it appears
that they treat violations of this contract as bugs to be fixed in
hardware or microcode, as evidenced by their fixes for Meltdown and
L1TF, described below.

%% The contract is aspirational, but this paper argues that it would
%% be good if processor vendors commit to \contract.  The contract
%% imposes few restrictions on the processor architecture: it allow
%% processors to continue aggressively use speculation, except for TLB
%% and page-table entries.  Yet, it provides software developers with a
%% mechanism to control what physical memory can leak (i.e., only memory
%% that has a virtual address in the current page table).  Furthermore,
%% because OS developers expect page faults and TLB misses to be quite
%% expensive (compared to memory references), \contract doesn't change
%% their performance expectations: developers already have adapted their
%% designs to avoid excessive page faults or TLB invalidations.

\paragraph{USC and attacks.}

The contract captures a common pattern that emerges in many transient
execution attacks: an adversary can only leak micro-architectural
state that is already present on the CPU, as well as the contents
of mapped memory, but not the contents of memory that is not
present in a page table.  As one example, consider the MDS family
of attacks~\cite{canella:fallout, schwarz:zombieload,schaik:ridl}.
These attacks allow an adversary to trick the kernel into leaking the
contents of mapped memory, through careful orchestration of transient
execution.  Linux prevents this class of attacks by clearing CPU buffers
when crossing the user-kernel boundary.  This is needed because, when
executing in kernel mode, all system memory is mapped and therefore
could be leaked through transient execution.  The contract, however,
captures the fact that only mapped memory is at risk with this attack.
This allows for a more efficient mitigation of such attacks, as we
demonstrate in \sys, by avoiding kernel mappings of sensitive memory.

In contrast to the example of MDS attacks, which leak sensitive data
from memory, the \contract{} does not help mitigate attacks that leak
sensitive data already present in the CPU state.  For instance, the
Spectre variant that leaks the contents of x86 MSRs (Spectre 3a) is not
precluded by the contract, since the sensitive data being leaked is not
present in memory at all.  As a result, an OS kernel must apply other
mitigations to deal with such attacks.

More generally, the contract helps categorize existing attacks
based on which part of the system state they leak, as shown in
\autoref{fig:attack-by-state}.  For attacks that leak core or uncore
state, the contract has little to say in terms of how those attacks
can be mitigated, as shown in the ``Mitigated by USC'' column.  As a
result, \sys defends against these attacks much in the same way as Linux.
In contrast, for attacks that leak the contents of memory, the contract
gives a more efficient mitigation approach: simply avoid mapping memory
that contains sensitive data.  This allows \sys to efficiently mitigate
attacks such as some variants of Spectre and MDS.

\begin{figure}
\small
\centering
\begin{tabular}{@{}llll@{}}
\textbf{Attack} & \textbf{Leaked state} & \textbf{Mitigated} & \textbf{Consistent} \\
&& \textbf{by USC} & \textbf{with USC} \\
\midrule

Spectre variants & \multirow{5}{0.75in}{Memory} & Yes & Yes\\
Meltdown & & Yes & Yes (ucode)\\
MDS & & Yes & Yes\\
PortSmash & & Yes & Yes\\
L1TF & & Yes & Yes (ucode)\\

\midrule

Spectre variants & \multirow{3}{0.75in}{Core state} & No & Yes \\
LazyFPU & & No & Yes \\
System reg. read & & No & Yes \\

\midrule

Spectre variants & \multirow{3}{0.75in}{Uncore state} & No & Yes \\
CrossTalk & & No & Yes \\
SGAxe & & No & Yes \\

\end{tabular}
\caption{Transient execution attacks categorized based on the
  state leaked by the attack.}
\label{fig:attack-by-state}
\end{figure}

As shown in the ``Consistent with USC'' column, all of the attacks
in \autoref{fig:attack-by-state} are consistent with the contract's
requirements on the underlying hardware.  This is good in two ways.
First off, this means that none of the known attacks violate the contract,
and thus, the contract is a reasonable approach for mitigating transient
execution attacks.  Second, this means that \contract can mitigate
the class of attacks that it covers---namely, attacks that leak memory
contents.

There are two special cases:
Meltdown and L1TF.  When originally discovered, these attacks bypassed
the page table protections and allowed an adversary to obtain the
contents of memory that was not mapped.  In both of these cases, the
hardware manufacturer (Intel) considered them to be hardware bugs, as
evidenced by the fact that both of them were fixed through hardware and
microcode revisions~\cite{intel:meltdown, intel:l1tf}, as confirmed by
Canella et al.~\cite{sok:transient}.\footnote{Canella et al. state that
some variants of the Meltdown attack, such as Meltdown-BR, are still
possible even with the most recent microcode.  Those variants, however,
are bypassing software checks, rather than the hardware page table, and
therefore do not violate the unmapped speculation contract.}
%% In contrast,
%% for other attacks like Spectre and MDS, the hardware manufacturer puts the
%% mitigation responsibility on the OS kernel, and the contract \emph{does}
%% hold for those attacks.


