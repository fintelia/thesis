%Your abstract text goes here. Just a few facts. Whet our appetites.
%Not more than 200 words, if possible, and preferably closer to 150.

Today's kernels pay a performance penalty for mitigations---such as
KPTI, retpoline, return stack stuffing, speculation barriers---to
protect against transient execution side-channel attacks such as
Meltdown~\cite{lipp:meltdown} and Spectre~\cite{kocher:spectre}.

%The
%\bench study identifies these security measures as one of the root causes
%of Linux slowdown~\cite{lebench}.

To address this performance penalty, this paper articulates the
\emph{unmapped speculation contract}, an observation that memory that
isn't mapped in a page table cannot be leaked through transient execution.
To demonstrate the value of this contract, the paper presents \sys,
a new kernel design that maintains a separate kernel page table for
every process.  This page table contains mappings for kernel
memory that is safe to expose to that process.
%, either by accessing it
%directly in kernel mode, or by leaking it through transient execution.
Because a process
doesn't map data of other processes, this design allows for many system
calls to execute without any mitigation overhead. When a process needs
access to sensitive data, \sys switches to a kernel page table that
provides access to all of memory and executes with all mitigations.

An evaluation of the \sys design implemented in the sv6 research
kernel~\cite{clements:sc} shows that \bench~\cite{lebench} can execute many system
calls without mitigations.  For some hardware generations, this results
in performance improvement ranging from a few percent (\texttt{huge page
fault}) to several factors (\texttt{getpid}), compared to a standard
design with mitigations.
