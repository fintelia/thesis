
\singlespace
\begin{center}

{\large \bf Understanding and Improving the Performance of Mitigating Transient Execution Attacks} \\[.5\baselineskip]
by \\
Jonathan Behrens \\[.5\baselineskip]
\end{center}

Submitted to the Department of Electrical Engineering and Computer Science on January 26, 2022 in Partial Fullfillment of the Requirements for the Degree of Doctor of Philosophy in Electrical Engineering and Computer Science.\\[.5\baselineskip]

\noindent
ABSTRACT \\

This these makes two contributions:
(1) it measures the performance evolution of mitigations against transient execution side channel attacks over several generations of processors, and (2) introduces the \sys kernel design, which eliminates as much as half the overhead on older processors.

The measurement study maps end-to-end overheads to the specific mitigations that cause them.
It reveals that hardware fixes for several transient execution attacks have reduced overheads on OS heavy workloads by a factor of ten.
However, overheads for JavaScript applications have remained roughly flat because they are caused by mitigations for attacks that even the most recent processors are still vulnerable to.
Finally, the study shows that a few mitigations account for most performance costs.

% On workloads that stress the Linux kernel interface (which have received the most attention) there have been substantial improvements with slowdowns on the LEBench benchmark suite going from over 30\% to less than 3\%.
% By contrast, the overhead for JavaScript applications running inside Firefox are impacted by an almost entirely different set of mitigations which haven't gotten better with newer processors.

\sys is a novel operating system architecture that is resilient to transient execution attacks, yet avoids many of the expensive software mitigations that existing operating systems employ when running on pre-2018 processors.
It leverages a new hardware/software contract termed the Unmapped Speculation Contract, which describes baseline limits on the speculative behavior of processors.

~\\[\baselineskip]

\noindent
Thesis  Supervisor:  M.  Frans  Kaashoek \\
Title:  Charles  Piper  Professor  of Electrical  Engineering  and  Computer  Science \\[.5\baselineskip]
\noindent
Thesis  Supervisor:  Adam Belay \\
Title:  Assistant Professor  of Electrical  Engineering  and  Computer  Science \\

\doublespace