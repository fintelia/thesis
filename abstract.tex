
\singlespace
\begin{center}

{\large \bf Performance Implications of Mitigating Transient Execution Side Channel Attacks} \\[.5\baselineskip]
by \\
Jonathan Behrens \\[.5\baselineskip]
\end{center}

Submitted to the Department of Electrical Engineering and Computer Science on January 7, 2022 in Partial Fullfillment of the Requirements for the Degree of Doctor of Philosophy in Electrical Engineering and Computer Science.\\[.5\baselineskip]

\noindent
ABSTRACT \\

This thesis quantifies how the performance overhead of mitigating transient execution attacks has evolved over subsequent generations of processors and presents an operating system design to reduce OS-level overheads on the processors where they are most severe.  

On workloads that stress the Linux kernel interface (which have received the most attention) there have been substantial improvements with slowdowns on the LEBench benchmark suite going from over 30\% to less than 3\%.
By contrast, the overhead for JavaScript applications running inside Firefox are impacted by an almost entirely different set of mitigations which haven't gotten better with newer processors.

\sys is a novel operating system architecture which is resilient to transient execution attacks, yet avoids many of the expensive software mitigations that existing operating systems employ when running on older processors. 
It achieves considerably better performance on many workloads compared to conventional operating system designs.

~\\[\baselineskip]

\noindent
Thesis  Supervisor:  M.  Frans  Kaashoek \\
Title:  Charles  Piper  Professor  of Electrical  Engineering  and  Computer  Science \\[.5\baselineskip]
\noindent
Thesis  Supervisor:  Adam Belay \\
Title:  Assistant Professor  of Electrical  Engineering  and  Computer  Science \\

\doublespace