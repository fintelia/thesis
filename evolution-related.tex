\section{Related Work}
\label{s:related}

The mitigations this work explores became necessary after the discovery of Meltdown~\cite{lipp:meltdown} and the original Spectre~\cite{kocher:spectre} variants.
These were rapidly followed by the discovery of more attacks targeting transient execution, including MDS~\cite{canella:fallout,schwarz:zombieload,schaik:ridl}, Speculative Store Bypass~\cite{horn:speculative-store-bypass}, and many others~\cite{bhattacharyya:smotherspectre,bulck:foreshadow,chen:sgxspectre,koruyeh:spectrersb,ragab:crosstalk,stecklina:lazyfp,schaik:cacheout,weisse:foreshadow-ng, bulck:lvi}.

%Flush Conflict~\cite{weber:osiris} automated discovery

Many of these attacks have been fixed in production hardware~\cite{intel:affected-processors}, while known hardware techniques for addressing others require more substantial changes~\cite{ainsworth:muontrap,yu:stt,yu:sdo}.

Since an enormous number of older processors are still in active use, software defenses are also essential.
For operating systems, these include KAISER~\cite{gruss:kaiser} and retpolines~\cite{intel:retpoline}.

User-space sandboxing requires its own set of techniques.
Swivel~\cite{narayan:swivel} is a compiler framework which hardens WASM bytecode against attack, while
Firefox's and Chrome's WASM engines rely on Site Isolation~\cite{reis:site-isolation}.
Production JavaScript engines deploy more targeted mitigations like Pointer Poisoning and Index Masking~\cite{webkit:spectre-meltdown}, and also reduce the overall timer precision~\cite{mozilla:timer-precision, webkit:spectre-meltdown}.
Compiler techniques like Speculative Load Hardening~\cite{carruth:slh} ensure binaries are completely immune to Spectre, albeit at considerable overhead.

Several survey papers~\cite{hill:survey,sok:transient,xiong:survey} classify attacks but not their performance impact, while LEBench~\cite{ren:lebench} explored how mitigations impacted performance across kernel versions.
There were a few early performance studies~\cite{nikolay:meltdown-spectre-performance,prout:measuring-spectre-meltdown} but they predate more recently discovered attacks.
The Linux community has paid close attention to the cost of mitigations including for IBRS~\cite{linus:ibrs-rant},  KPTI~\cite{gregg:kpti-perfromance}, and MDS~\cite{larabel:perf-zombieload} which has played a role in both understanding and driving down the performance overheads.

Where this work stands out is by looking at many different generations of CPUs and characterizing the performance impacts of the set of mitigations deployed in practice.

% Uses size information to avoid spectre
% https://www.cs.columbia.edu/~mtarek/files/preprint_ISCA21_NoFAT.pdf

% MuonTrap. A hardware approach to preventing spectre attacks by preventing speculative accesses from ever entering the cache. 5% slowdown on SPEC, but actually a speedup on PARSEC due to their tiny but fast L0 cache.
% https://arxiv.org/pdf/1911.08384.pdf

% Speculative Data-Oblivious Execution. Hardware approach to preventing spectre attack. Builds on Speculative Taint Tracking.
% [SDO] https://iacoma.cs.uiuc.edu/iacoma-papers/isca20_2.pdf
% [STT] https://www.cs.tau.ac.il/~mad/publications/micro2019-stt.pdf
%
% Contains a good summary of related work on hardware defenses for speculative execution.

% Mitigation list on Intel Processors.
% https://software.intel.com/content/www/us/en/develop/topics/software-security-guidance/processors-affected-consolidated-product-cpu-model.html

% Swivel: Hardening WebAssembly against Spectre
% https://www.usenix.org/system/files/sec21-narayan.pdf

% DOLMA: Hardware approach with 10-40% but more complete coverage of attacks, including those targeting registers
% https://www.usenix.org/conference/usenixsecurity21/presentation/loughlin

% Rage Against the Machine Clear. Study of machine clears as a root cause for attacks. Also has a good list of attacks in related work section (plus list 3 survey papers on it)
% https://www.usenix.org/system/files/sec21-ragab.pdf
