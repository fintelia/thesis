%\section{Conclusion}
\label{s:concl}

This paper articulates the unmapped speculation contract (\contract)
for a division of labor between hardware and software.  This contract
allows hardware to speculate on many values (but not the values of
page table entries) and provides software with a mechanism to prevent
leaking secrets through micro-architectural state.  The \sys design
shows how \contract can be used to reduce the performance costs of
mitigations on system calls using per-process Q domains and global K
domains.  \sys transparently switches from Q- to K-domain through page
faults, uses temporary mappings to access unmapped physical pages, and
splits data structures into public and private parts.  An evaluation
shows that \sys can run the microbenchmarks of \bench with small
performance overhead compared to a kernel without mitigations: for
18 out of 30 \bench microbenchmarks, \sys's performance is within 5\%
of the performance without mitigations
%% , and in the worst case it is
%% $2\times$ instead of $7\times$ (but at the cost of some extra memory
%% overhead)
.  Although \sys is research kernel, we are hopeful that its
ideas can carry over to production monolithic kernels.
