\section{Implementation}
\label{s:impl}

\begin{figure*}
\centering
\small
\begin{tabular}{lll}
{\bf Transient execution variant} & {\bf Strategy} & {\bf Support} \\
\midrule
V1 (Bound Check Bypass) & bounds clipping & partial \\
V1.1 (Bounds Check Bypass Store) & lfence & partial \\
V1.2 (Read-only Protection Bypass) & lfence & n/a (no in-kernel software sandbox) \\
V2 (Branch Target Injection) &	retpoline & yes \\
\quad \ditto & speculation barrier	&	yes \\
\quad \ditto & return stack buffer filling	&	yes \\
\quad \ditto & disable spec before BIOS calls &	n/a (no calls to BIOS in \sys) \\
V3 (Meltdown) & Kernel page table isolation &	yes \\
V3a (System Register Read) & microcode & yes \\
V4 (Speculative Store Bypass) &	disable spec. or ctx. switch & yes \\
LazyFPU	& hardware-assisted save/restore & yes \\
SpectreRSB & return stack buffer filling & yes \\
L1TF &	cache flush, no SMT & 	n/a (no VM entry in \sys)  \\
\quad \ditto & no invalid PTEs	&	yes \\
PortSmash & no SMT  &	no \\
Microarchitectural Data Sampling  & CPU buffer clearing & yes \\
(Fallout, RIDL, Zombie Load, etc.) & no SMT &	no \\
Load Value Injection & lfence &	n/a  (no SGX in \sys) \\
Meltdown-PK 
(protection key bypass)
& address space isolation & n/a (no protection keys) \\
Meltdown-BR
(bounds check instr.)
& lfence & n/a (no bounds check instructions) \\
\end{tabular}
\caption{Transient execution mitigations implemented in \sys.}
\label{fig:mitigations-impl}
\end{figure*}

To demonstrate the feasibility of the \sys design, we implemented a
prototype of \sys starting from the sv6 research kernel. The kernel
  is monolithic, implementing traditional OS services such as virtual
  memory, processes and threads, file systems, fine-grained
  concurrency using RCU-like techniques, etc.  The sv6 kernel, is
  written in C/C++, runs on x86 processors (both AMD and Intel), and
  has decent uniprocessor performance and great multicore performance
  and scalability~\cite{clements:sc}.

\paragraph{Kernel changes.}

\sys's design affects most core kernel subsystems, including the memory
allocator, virtual memory, context switching and the scheduler, and the
file system.  The simplicity of sv6 allowed for rapid experimentation
with kernel designs to enable \sys, which would have been challenging
to do in a more complex kernel like Linux, since it is time-consuming
to make changes to core subsystems in the Linux kernel, which would have made
design iterations far slower.

To help partition the kernel data structures across Q domains, we
developed Warden, a tool for tracking down the cause of world switches.
Warden instruments page faults from the Q domain that lead to a
world switch, and records a stack trace for each of them.  Examining the
profile of these world switches allows the kernel developer to quickly
understand what kernel data structures need to be partitioned or sharded
to reduce the number of world switches, as well as the operations
that need to be supported on these data structures within a Q domain.
Although Warden identifies the data structures that are causing world
switches, it is up to the kernel developer to identify an appropriate
plan for partitioning the data structure so that no sensitive data can
leak through side channels.

To run applications on top of the \sys prototype kernel, we changed the
\sys system call interface, including system call numbers, data structure
layout, etc, to match that of Linux.  This allows unmodified Linux
ELF executables to run on top of \sys, and ensures that \sys implements
(a subset of) the same system calls that are available on Linux.

We modified sv6 to use PCIDs to reduce the cost of switching page
tables (see \autoref{ss:switch}).  To improve TLB shootdown
performance, we modified sv6 to use Linux's shootdown strategy.  This
is important, for example, for removing temporary mappings in a
\texttt{read} and \texttt{write} systems calls (see \autoref{ss:fs}).

\paragraph{Mitigations.}

\sys implements side-channel mitigations for known transient
execution attacks~\cite{hill:survey,sok:transient}, as shown
in \autoref{fig:mitigations-impl}.  \sys mostly copies the
mitigation strategies and their implementation from the Linux
kernel~\cite{linux:vuln}; the most interesting exception is that \sys
does not apply some of these mitigations to the Q domain, as described
in \autoref{fig:mitigations}.

For Spectre V1, \sys, adds an \texttt{lfence} instruction when copying
from user code, and when taking an interrupt, exception, and NMI entry.
\sys uses bounds clipping in fewer cases than Linux for two reasons:
\sys has less code and we haven't performed a careful audit of the
complete source code.  For Spectre V2, we compile \sys to use retpolines
(by specifying the ``-mretpoline-external-thunk'' flag to \texttt{clang}). \sys
also uses Linux's \texttt{FILL\_RETURN\_BUFFER} macro to fill the
return stack buffer, and issues an indirect branch predictor barrier
\texttt{IBPB} instruction on a context switch. For Meltdown, \sys uses
separate page tables (as described in \autoref{ss:overview}) and uses
process-context identifiers (\texttt{PCID}s) to avoid TLB flushes.

For Spectre V4, \sys issues an \texttt{lfence} on context switch. (If
\sys supported generating code at runtime, the JITs would also have to
be hardened.)  For LazyFP, \sys uses the \texttt{xsaveopt}
instruction to safe/restore floating point state.  For SpectreRSB,
\sys fills the return stack buffer on context switch.  For L1TF, \sys
avoids invalid PTEs. Like Linux, \sys doesn't address PortSmash; the
default for the Linux kernel is to allow SMT, and \sys does too.  For
microarchitectural data sampling attacks, \sys issues the
\texttt{verw} instruction for clearing CPU buffers.

Some attacks aren't applicable to \sys, because \sys doesn't support
virtualization, secure enclaves, and hardware transactional memory;
does not call into the BIOS; and does not implement in-kernel software
sandboxes such as BPF.

Like Linux, \sys also zeroes unused CPU registers on kernel entry, to
reduce the avenues of attack available to an adversary.  To determine
whether mitigations are necessary, \sys maintains a special variable
called \texttt{secrets\_mapped} whose value is 0 in the Q domain and 1 in
the K domain; this allows the rest of the kernel code to determine if it
needs to perform mitigations just by using \texttt{if (secrets\_mapped)
...} (as long as interrupts are disabled, to avoid races).  To help
evaluate the performance impact of side-channel mitigations, \sys's
implementation allows switching individual mitigations on and off at
runtime, rather than at compile time or boot time.

To improve performance, a few system calls invoke the world switch
intentionally to avoid the extra overhead of a transparent world switch.
For example, \texttt{open}, and \texttt{fork} always
invoke world switch intentionally.  The \texttt{read} and \texttt{write}
system calls invoke a world switch intentionally when they are reading
or writing large amounts of data, since the cost of a world switch is
less than the cost of shooting down the temporary mappings for that many
file pages.  A page fault on a Copy-On-Write (COW) page also intentionally
invokes a world switch.


\paragraph{Lines of code.}

The \sys prototype consists of about 34,000 lines of C++ code (for
\texttt{kernel/} and \texttt{include/}), compared to 24,000 lines of
C++ code for the sv6 kernel that \sys was derived from.
\texttt{git diff --stat} reports roughly 17,000 lines of insertions and 5,000
lines of deletions between sv6 and \sys.
It is difficult to further
break down \sys's lines of code, since many aspects of \sys's design
required small changes throughout the kernel's source code.  For
example, splitting up the kernel memory allocator required the use of
C++ placement \texttt{new} in many parts of the kernel.  Similarly,
implementing the Linux binary compatibility layer required making
changes to the implementation of many system calls.

