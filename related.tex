\label{s:related}

This thesis is motivated by the papers that show how secret kernel data
can be leaked through micro-architectural state, which started with the discovery of Meltdown~\cite{lipp:meltdown} and the original Spectre~\cite{kocher:spectre} variants.
These were rapidly followed by the discovery of more attacks targeting transient execution, including MDS~\cite{canella:fallout,schwarz:zombieload,schaik:ridl}, Speculative Store Bypass~\cite{horn:speculative-store-bypass}, and many others~\cite{bhattacharyya:smotherspectre,bulck:foreshadow,chen:sgxspectre,koruyeh:spectrersb,ragab:crosstalk,stecklina:lazyfp,schaik:cacheout,weisse:foreshadow-ng, bulck:lvi}.
Several survey papers were helpful by categorizing the known attacks~\cite{hill:survey,sok:transient,xiong:survey}.

Phoronix has conducted some of the most notable performance studies~\cite{phoronix:perf-zombieload, phoronix:two-years, phoronix:three-years}. A few other groups have conducted studies~\cite{nikolay:meltdown-spectre-performance,prout:measuring-spectre-meltdown} but those predate the most recent attacks.
The Linux community has paid close attention to the cost of mitigations throughout, including for IBRS~\cite{linus:ibrs-rant},  KPTI~\cite{gregg:kpti-perfromance}, and MDS~\cite{phoronix:perf-zombieload}.
Their work has played a role in both understanding and driving down the performance overheads.

\section{Mitigation Approaches}

Linux makes heavy use of software/microcode mitigations on older processors, but many attacks have now been fixed in production hardware~\cite{intel:affected-processors}.
Meanwhile known hardware techniques for addressing other attacks require more substantial changes~\cite{barber:specshield, weisse:nda, ainsworth:muontrap,yu:stt,yu:sdo}, generally by somehow delaying the use of speculative data until it is safe.
While such defenses are
more comprehensive, they have higher overheads that impact performance whenever
speculation occurs.
By contrast, the \contract constrains speculation in a more
targeted way based on memory mappings. ConTExT also proposes constraining
speculation based on memory mappings, but introduces a new PTE bit to explicitly
mark pages that contain secret data~\cite{ConTExT}. \sys instead keeps secrets in
separate address spaces, and allows speculation after employing its defenses to
switch to the K domain.  Finally, SpecCFI proposes to enforce control-flow
integrity during speculative execution~\cite{koruyeh:speccfi}. This idea strengthens
Spectre defenses, and is complementary to \sys.

User-space sandboxing requires its own set of techniques.
Swivel~\cite{narayan:swivel} is a compiler framework which hardens WASM bytecode against attack, while
Firefox's and Chrome's WASM engines rely on Site Isolation~\cite{reis:site-isolation}.
Production JavaScript engines deploy more targeted mitigations like Pointer Poisoning and Index Masking~\cite{webkit:spectre-meltdown}, and also reduce the overall timer precision~\cite{mozilla:timer-precision, webkit:spectre-meltdown}.
Compiler techniques like Speculative Load Hardening~\cite{carruth:slh} ensure binaries are completely immune to Spectre, albeit at considerable overhead.

\section{\sys}

We rely heavily on the mitigation work in the Linux
community~\cite{linux:vuln}. \sys adopts Linux's techniques and their
optimized implementation in the K domain.  \sys uses, for example,
Linux's \texttt{nospec} macro for bounds clipping,
\texttt{FILL\_RETURN\_BUFFER} to fill the return buffer, and retpoline.
\sys's hotpatching of its kernel text to remove retpolines in the Q
domain was inspired by Linux's \textsc{alternative}
macro~\cite{lwn:alternative}.


The Q page table is inspired by the shadow page table in
KAISER~\cite{gruss:kaiser} and KPTI~\cite{linux:kpti}. In Linux, when
a process executes in user space, the process runs with a shadow
page table, which maps only minimal parts of kernel memory: the kernel
memory to enter/exit the kernel on a system call. As soon as the process
enters the kernel, it switches to the kernel page table that maps all
of physical memory.  \sys, however, executes complete system
calls while running under the Q page table; this requires a significant
redesign of the OS kernel, which is a major focus of this work.

The use of virtual-memory to partition the kernel address space has a
long history in operating systems research.  One example is
Nooks~\cite{swift:nooks-tocs}, which runs device drivers in separate
protection domains with their own page table in kernel space to
provide fault isolation between drivers and the kernel.  Another
example is the use of Mondrian Memory Protection~\cite{witchel:mmp} to
isolate Linux kernel modules in different protection domains within
the kernel address space~\cite{witchel:mondrix}.  The most
recent example is Mike Rapoport's work on kernel address space
isolation~\cite{lwn:beyond-kpti} in Linux.  These designs use similar
techniques to introduce isolation domains within the kernel, but focus
on traditional attacks (e.g., code execution through a buffer overflow)
as opposed to transient execution.

%Flush Conflict~\cite{weber:osiris} automated discovery

% Several survey papers~\cite{hill:survey,sok:transient,xiong:survey} classify attacks but not their performance impact, while LEBench~\cite{ren:lebench} explored how mitigations impacted performance across kernel versions.

% Where this work stands out is by looking at many different generations of CPUs and characterizing the performance impacts of the set of mitigations deployed in practice.

% Uses size information to avoid spectre
% https://www.cs.columbia.edu/~mtarek/files/preprint_ISCA21_NoFAT.pdf

% MuonTrap. A hardware approach to preventing spectre attacks by preventing speculative accesses from ever entering the cache. 5% slowdown on SPEC, but actually a speedup on PARSEC due to their tiny but fast L0 cache.
% https://arxiv.org/pdf/1911.08384.pdf

% Speculative Data-Oblivious Execution. Hardware approach to preventing spectre attack. Builds on Speculative Taint Tracking.
% [SDO] https://iacoma.cs.uiuc.edu/iacoma-papers/isca20_2.pdf
% [STT] https://www.cs.tau.ac.il/~mad/publications/micro2019-stt.pdf
%
% Contains a good summary of related work on hardware defenses for speculative execution.

% Mitigation list on Intel Processors.
% https://software.intel.com/content/www/us/en/develop/topics/software-security-guidance/processors-affected-consolidated-product-cpu-model.html

% Swivel: Hardening WebAssembly against Spectre
% https://www.usenix.org/system/files/sec21-narayan.pdf

% DOLMA: Hardware approach with 10-40% but more complete coverage of attacks, including those targeting registers
% https://www.usenix.org/conference/usenixsecurity21/presentation/loughlin

% Rage Against the Machine Clear. Study of machine clears as a root cause for attacks. Also has a good list of attacks in related work section (plus list 3 survey papers on it)
% https://www.usenix.org/system/files/sec21-ragab.pdf


%% A kernel-organizational view of \sys is that the K domain is a
%% microkernel that provides services to Q domains~\cite{liedtke:sosp95},
%% but all running in kernel mode. Alternatively, one can view the K
%% domain as a hypervisor multiplexing several virtual machines that each
%% run an application with its Q domain~\cite{bugnion:disco,barham:xen}.
%% In both views, the interface between the Q and the K domains is more
%% porous than a microkernel or hypervisor interface: parts of data
%% structures of the K domain are directly mapped into Q domains to avoid
%% world switches.

%% other related work:

%% %https://www.microsoft.com/en-us/research/wp-content/uploads/2017/08/cloak_sec17.pdf

%% %https://www.usenix.org/system/files/conference/usenixsecurity18/sec18-dong.pdf

%% T-SGX (Shih, NDSS 2017)

%% architectural changes (e.g., removing side channel, avoiding resource
%% sharing)

%% Yan, InvisiSpec

%% oo7: low-overhead defense against spectre attacks via binary analysis

