\section{Attacks and Mitigations}
\label{s:background}

We consider attacks from the the perspective of how they affect end-user performance.
%Our interest is in what mitigations are used by default to maintain an acceptable level of security and what performance impact they cause.
This outlook differs from prior surveys like Canella, et al.~\cite{sok:transient} which focus on enumerating and classifying the space of possible attacks.

\subsection{Meltdown-Type Attacks}

Meltdown-type attacks exploit the processor's fault-handling logic to speculatively access privileged state.

\subsubsection{Meltdown}
The original Meltdown attack is caused by speculatively translating kernel addresses for supervisor pages even while running in user mode, which enables a user process to read any kernel memory mapped into its address space.
Processors from Intel as well as some from IBM and ARM were vulnerable~\cite{intel:meltdown,ibm:speculation, arm:speculation}.

Software mitigations for Meltdown are expensive, requiring a page table
switch on every user-kernel boundary crossing.  Processors made by other
vendors---and those designed after the attack was discovered---do not engage in
this kind of speculation, so they can avoid these software overheads.

\subsubsection{L1 Terminal Fault}
%L1 Terminal fault is a pretty clear case of a hardware bug.
On certain Intel processors, the present bit in PTEs is ignored during speculative execution, which can
allow an attacker to leak L1 cache contents.
Operating system software can easily be adjusted to make sure no vulnerable PTEs are included in the page tables, which mitigates the attack at virtually zero cost.

However, when running a hypervisor, the same speculative mechanisms also bypass
the nested page table.
Taken together, if the hypervisor doesn't flush the L1 cache before every VM entry, it risks leaking recently accessed data from other privilege domains.
Both the flush operation itself and subsequent cache misses make this mitigation more costly.

\subsubsection{LazyFP}
Traditionally, when exposing floating point hardware to user processes, operating systems would optimize context switch time by lazily saving and restoring FPU state.
In particular, the assumption was that many processes would not access floating point state, so on a context switch the FPU would be marked disabled, but retain the floating point registers from the previously running process.
Any attempt to execute floating point instructions would trigger a trap, during which the OS could save the old process's floating point registers and load in the registers for the current process.

During transient execution, some processors will ignore the enable bit on the FPU and allow computation on the floating point registers even if they actually belong to a different process, potentially leaking sensitive register contents to an attacker.

Linux's mitigates LazyFP by always saving and restoring FPU state during context
switches.  Amusingly, this mitigation actually speeds up certain workloads,
because modern processors provide highly efficient instructions for saving and
restoring this state (e.g., \texttt{xsaveopt})~\cite{lutomirski:lazyfp-perf}. As
a result, the trap handling overhead of LazyFP is often higher.

% \subsubsection{System Register Bypass}
% Certain processors will also speculatively let user space code execute reads of system registers.
% Because there is no kernel intervention at all, only a microcode patch can prevent this attack.

\subsection{Spectre-Type Attacks}

Spectre-type attacks exploit speculative execution following a misprediction.
They typically involve a specific gadget involving two memory loads; see Figure~\ref{fig:spectre-gadget} for an example.
The first load brings sensitive data into a register, and the second uses that value to index into a large array.
An attacker is later able to determine the result of the first load by seeing which array entry was pulled into the cache.

\begin{figure}[h]
\begin{lstlisting}
int x = array[index];
int y = array2[x * 256];
\end{lstlisting}
\caption{A Spectre gadget. If this code sequence is executed (even speculatively) it will alter the contents of the CPU cache, making it possible for an attacker to learn the value of \texttt{x}, or if \texttt{index} is attacker controlled, all of virtual memory. }
\label{fig:spectre-gadget}
\end{figure}

\subsubsection{Spectre V1}
The bounds-check-bypass variant of Spectre works by tricking the processor into doing an out-of-bounds array access by speculatively executing the body of an if statement.
Software mitigations usually entail manually annotating kernel branches with \texttt{lfence} instructions or array accesses with special macros that never read out of bounds.

\subsubsection{Spectre V2}

Modern CPUs use a Branch Target Buffer to predict the targets of indirect branches so they can speculatively start executing them before resolving the true target of a branch.
Poisoning the branch target buffer enables attacker code to make the CPU mispredict the targets of indirect branches and route transient execution to specially chosen spectre gadgets.

There is no single mitigation for Spectre V2.
It is commonly mitigated by replacing every indirect branch with a retpoline sequence \cite{intel:retpoline} that halts further speculative execution plus additional kernel logic to protect user processes from one another.

\subsubsection{Speculative Store Bypass}
This attack---also known as Spectre V4---exploits store-to-load forwarding in modern processors to learn the contents of recently written memory locations.
The only available mitigation is a processor mode called Speculative Store Bypass Disable (SSBD), but enabling it has severe negative performance impacts.
Given the difficulty of exploiting Speculative Store Bypass and the considerable cost of SSBD, by default the mitigation is only used by Linux for processes that specifically opt in to it via \texttt{prctl} or \texttt{seccomp}.

\subsection{Microarchitectural Data Sampling (MDS)}
Microarchitectural Data Sampling describes class of attacks involving leaks from various microarchitectural buffers within the CPU.
Unlike other Spectre and Meltdown variants, MDS attacks cannot be targeted to specific victim addresses which makes them more challenging to exploit.

From an attacker perspective there are many different variations of MDS with their own specific mechanisms and capabilities.
However, mitigations all fall into two categories: specific microarchitectural buffers need to be cleared on every privilege domain crossing or hyperthreading must be disabled to prevent an attacker and victim from simultaneously running on the same physical core.
Clearing these CPU buffers is costly because of how frequently it must be done.
Not using hyperthreading would have an even larger cost, but by default
hyperthreading is enabled even for vulnerable CPUs because the risk was viewed acceptable given the performance difference.

% Mitigating MDS requires a combination of microcode and software.
% The microcode adds new functionality to clear the relevant microarchitectural buffers, and software triggers the clears whenever it switches privilege domains.

% More troublingly, MDS can also leak information between hyperthreads of the same core.

% Preventing this requires either disabling hyperthreading (a substantial performance cost for most workloads) or pinning mutually distrusting tasks to disjoint hyperthread pairs.
