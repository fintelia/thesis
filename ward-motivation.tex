\section{Motivation}
\label{s:motivation}

\begin{figure*}[t]
  \begin{center}
    \includegraphics{results/bhw2_bhw3_overhead}
  \end{center}
\vspace{-\baselineskip}
\caption{Linux slowdown due to mitigations on \bench, for two generations of Intel CPUs: Skylake and Cascade Lake.}
\label{fig:linuxslowdown}
\end{figure*}

\begin{figure*}[t]
  \begin{center}
    \includegraphics{results/cascade_lake_regression}
  \end{center}
\vspace{-\baselineskip}
\caption{Performance regression on the newer Cascade Lake CPU, compared to the older Skylake CPU,
  for \bench on Linux, with all software-controllable mitigations disabled.}
\label{fig:regression}
\end{figure*}

Transient execution mitigations harm kernel performance in two
ways. First, they place overhead on code execution by disabling
speculation.  For example, the Linux Kernel uses a retpoline patch to
mitigate Spectre V2, which replaces each indirect branch with a
sequence of instructions that prevent the CPU from performing branch
target speculation~\cite{intel:retpoline}. Second, these mitigations
increase the privilege mode switching cost incurred during each system
call: upon entry into the kernel, they either flush microarchitectural
state or reconfigure protection mechanisms.  For example,
KPTI~\cite{gruss:kaiser,linux:kpti} switches to a separate page table
before executing kernel code to prevent Meltdown
attacks~\cite{lipp:meltdown}.  Workloads
that are system call intensive (e.g., web servers, version control
systems, etc.) are impacted by this type of overhead, while non-kernel
intensive workloads see little performance impact~\cite{gruss:kaiser}.

Collectively, these and other mitigations can result in large
slowdowns. To better understand this problem, we run
\bench~\cite{lebench}, a microbenchmark suite of system calls
that impact application
performance the most.  We evaluate the Linux kernel (version 5.6.13),
comparing two configurations: one where all mitigations are disabled
and one where all are enabled. Figure~\ref{fig:linuxslowdown} shows
the relative slowdown between the two configurations for 13 kernel
operations of \bench that don't involve networking (i.e., without
\texttt{send}, \texttt{recv}, \texttt{epoll}).  There are two sets of
bars, representing two generations of Intel CPUs: the older Skylake,
and the newer Cascade Lake.  On the older Skylake CPUs,
system calls that perform the least kernel work are impacted the most
(e.g., \texttt{getpid()}), but a wide range of operations are impacted
significantly (25\%-100\% slowdowns). These observations are similar
to those made by Ren et. al.; they find that KPTI and Spectre V2
mitigations are the root cause of slowdowns in the Linux
Kernel over the last two years~\cite{lebench}.

The newer Cascade Lake CPUs exhibit lower relative overheads, partly
because the processors include hardware mitigations for some of the
transient execution vulnerabilities.  However, these lower overheads
are also in part due to the newer Cascade Lake CPUs being \emph{slower}
in the baseline case when software-controllable mitigations are disabled.
Figure~\ref{fig:regression} shows the performance of the microbenchmark
on Cascade Lake (Intel Xeon Silver 4210R) relative to the earlier
Skylake CPU (Intel Xeon E5-2640 v4).  Our experiment uses CPUs
with identical clocks (2.4~GHz), and nearly identical other hardware
(Dell PowerEdge T430 vs. T440), which allows the comparison to be
meaningful.  The results demonstrate that, although the new CPU is
faster at some microbenchmarks, it is slower for many others: e.g.,
context-switching is about 20\% slower.  Although it is impossible for us to
separate slowdowns due to mitigations from speedups due to architectural
improvements, the results suggest that the overheads of mitigations implemented
in hardware (e.g., for Meltdown, L1TF, or MDS) could still be significant.
\footnote{One indication that this regression may be related to hardware
  mitigations is that measured branch mispredictions are around 40\% higher on
  LEBench.}

%\fk{table 11 in ~\cite{sok:transient} claims the overhead of KPTI is
%  small, < 3\%}

%Avoiding these overheads normally requires making compromises to
%security; disabling any one of them would leave the kernel vulnerable
%to a transient execution attack.
%Our goal, instead, is to apply these
%mitigations only when they are needed to protect data that should be
%kept secret from the running process.
%We now describe our approach to
%restructuring the kernel so it can achieve this goal.
