\section*{Related work}

This paper is motivated by the papers that show how secret kernel data
can be leaked through micro-architectural state
(e.g.,~\cite{lipp:meltdown, kocher:spectre, bulck:foreshadow,
  schaik:ridl,canella:fallout,schwarz:zombieload}).  In particular,
two survey papers were helpful by categorizing the known
attacks~\cite{hill:survey,sok:transient}.

This paper relies heavily on the mitigation work in the Linux
community~\cite{linux:vuln}. \sys adopts Linux's techniques and their
optimized implementation in the K domain.  \sys uses, for example,
Linux's \texttt{nospec} macro for bounds clipping,
\texttt{FILL\_RETURN\_BUFFER} to fill the return buffer, and retpoline.
\sys's hotpatching of its kernel text to remove retpolines in the Q
domain was inspired by Linux's \textsc{alternative}
macro~\cite{lwn:alternative}.

In addition to the software/microcode approach currently used by Linux
and other production operating systems, there are several proposed
hardware-only defenses that delay the use of speculative data until
it is safe~\cite{barber:specshield, weisse:nda, yu:stt}. While these defenses are
more comprehensive, they have higher overheads that impact performance whenever
speculation occurs. By contrast, the \contract constrains speculation in a more
targeted way based on memory mappings. ConTExT also proposes constraining
speculation based on memory mappings, but introduces a new PTE bit to explicitly
mark pages that contain secret data~\cite{ConTExT}. \sys instead keeps secrets in
separate address spaces, and allows speculation after employing its defenses to
switch to the K domain.  Finally, SpecCFI proposes to enforce control-flow
integrity during speculative execution~\cite{koruyeh:speccfi}. This idea strengthens
Spectre defenses, and is complementary to \sys. 

The Q page table is inspired by the shadow page table in
KAISER~\cite{gruss:kaiser} and KPTI~\cite{linux:kpti}. In Linux, when
a process executes in user space, the process runs with a shadow
page table, which maps only minimal parts of kernel memory: the kernel
memory to enter/exit the kernel on a system call. As soon as the process
enters the kernel, it switches to the kernel page table that maps all
of physical memory.  \sys, however, executes complete system
calls while running under the Q page table; this requires a significant
redesign of the OS kernel, which is a major focus of this paper.

The use of virtual-memory to partition the kernel address space has a
long history in operating systems research.  One example is
Nooks~\cite{swift:nooks-tocs}, which runs device drivers in separate
protection domains with their own page table in kernel space to
provide fault isolation between drivers and the kernel.  Another
example is the use of Mondrian Memory Protection~\cite{witchel:mmp} to
isolate Linux kernel modules in different protection domains within
the kernel address space~\cite{witchel:mondrix}.  The most
recent example is Mike Rapoport's work on kernel address space
isolation~\cite{lwn:beyond-kpti} in Linux.  These designs use similar
techniques to introduce isolation domains within the kernel, but focus
on traditional attacks (e.g., code execution through a buffer overflow)
as opposed to transient execution.

%% A kernel-organizational view of \sys is that the K domain is a
%% microkernel that provides services to Q domains~\cite{liedtke:sosp95},
%% but all running in kernel mode. Alternatively, one can view the K
%% domain as a hypervisor multiplexing several virtual machines that each
%% run an application with its Q domain~\cite{bugnion:disco,barham:xen}.
%% In both views, the interface between the Q and the K domains is more
%% porous than a microkernel or hypervisor interface: parts of data
%% structures of the K domain are directly mapped into Q domains to avoid
%% world switches.

%% other related work:

%% %https://www.microsoft.com/en-us/research/wp-content/uploads/2017/08/cloak_sec17.pdf

%% %https://www.usenix.org/system/files/conference/usenixsecurity18/sec18-dong.pdf

%% T-SGX (Shih, NDSS 2017)

%% architectural changes (e.g., removing side channel, avoiding resource
%% sharing)

%% Yan, InvisiSpec

%% oo7: low-overhead defense against spectre attacks via binary analysis

