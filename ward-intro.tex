%\section{Introduction}

Over the last two years, transient execution has emerged as a
powerful new side-channel attack technique.  Vulnerabilities have
proliferated~\cite{sok:transient,hill:survey}, with examples now including Meltdown~\cite{lipp:meltdown},
Spectre~\cite{kocher:spectre}, L1 Terminal Fault~\cite{bulck:foreshadow}, RIDL~\cite{schaik:ridl},
Fallout~\cite{canella:fallout}, ZombieLoad~\cite{schwarz:zombieload},
CrossTalk~\cite{ragab:crosstalk}, and SGAxe~\cite{sgaxe}.  In contrast
with conventional timing-based side-channel attacks~\cite{kocher:timing},
where the victim must access its data in a specific pattern in order
to leak it, transient execution attacks are more serious because they
often allow an attacker to precisely control which memory locations are
leaked, including memory that might not be accessed on the committed
execution path.  This is of particular concern to OS kernels, which have
access to all of physical memory, and therefore could leak data from any
process through transient execution bugs.  In a public cloud,
where it is common for mutually distrustful tenants to share
a single machine~\cite{borg, cpi2}, the threat of transient execution
is especially concerning.

A key challenge in addressing transient execution attacks lies in
minimizing the performance overheads.  CPU and OS designers have
implemented a range of mitigations to defeat transient execution
attacks, including state flushing, selectively preventing speculative
execution, and removing observation channels~\cite{sok:transient}.
These mitigations impose performance overheads (see
\autoref{s:motivation}): some of the mitigations must be applied at
each privilege mode transition (e.g., system call entry and exit), and
some must be applied to all running code (e.g., retpolines for all
indirect jumps).  In some cases, they are so expensive that OS vendors
have decided to leave them disabled by default~\cite{apple_support,
  microsoft_support}.  Recent processor designs have also incorporated
mitigations into hardware, which also reduces performance compared to
earlier processor designs that do not perform such hardware
mitigations.

%Linux has considered offering
%an option to flush the entire L1 cache on context switch, to protect
%against future transient execution attacks~\cite{linux:vuln,phoronix:5.8}.

% For example,
% microarchitectural data sampling (MDS) vulnurabilities like RIDL and
% Fallout cannot be fully mitigated without turning off hyperthreading,
% reducing overall CPU performance by around 33\%.

To address the above challenge, this paper proposes a new
hardware/software contract, called the \emph{unmapped speculation
contract}, or \contract{} for short.  \contract{} allows the OS kernel
to significantly reduce the overhead of mitigating a particular subset
of transient execution attacks---namely, those that leak arbitrary
memory contents.  The \contract{} says that physical memory that is
unmapped (i.e., physical memory that has no virtual address) cannot
be accessed speculatively.  The
benefit of \contract is two-fold.  From the OS designer perspective,
it provides bounds on what data can be leaked through transient execution,
and, as we show in the rest of this paper, can significantly reduce the
cost of mitigations.  From the hardware designer perspective, \contract
allows the CPU to keep many of the current speculative execution
optimizations and their associated performance benefits.  Most
processor architectures already adhere to \contract; AMD
states that ``AMD processors are designed to not speculate into memory
that is not valid in the current virtual address memory range defined
by the software defined page tables''~\cite[pg. 2]{amd:speculation},
and Intel issued hardware and microcode fixes for bugs that violate
\contract~\cite{intel:meltdown, intel:l1tf}.

To demonstrate the benefits of the unmapped speculation contract,
this paper presents \sys{}, a novel kernel architecture that uses
selective kernel memory mapping to avoid the costs of transient execution
mitigations.  \sys{} maintains separate kernel memory mappings for each
process, and ensures that the memory mapped in the kernel of a process
does not contain any data that must be kept secret from that process.
As a result, privilege mode switches (e.g., system call entry and exit)
no longer need to employ expensive mitigations, since there are no
secrets that could be leaked by transient execution.  When the \sys{}
kernel must perform operations that require access to unmapped parts
of kernel memory, such as opening a shared file or context-switching
between processes, it explicitly changes kernel memory mappings, and
invokes the same mitigation techniques used by the Linux kernel today.

A key challenge in the \sys design lies in re-architecting the kernel
and its data structures to allow for per-process views of the
kernel address space.  For example, a typical \texttt{proc} structure in
the kernel contains sensitive fields, such as the saved registers of that
process, which should not be leaked to other processes.  At the same time,
every process must be able to invoke the scheduler, which in turn may
need to traverse the list of \texttt{proc} structures on the run queue.
This paper presents several techniques to partition the kernel:
transparent switching of kernel address spaces when accessing sensitive
pages through page faults; using temporary mappings to access unmapped
physical pages; splitting data structures into public and private
parts; etc.

To evaluate the \sys design, we applied it to the sv6 research
kernel~\cite{clements:sc} running on x86 processors.  The sv6 kernel
is a monolithic OS kernel written in C/C++, providing a POSIX
interface similar to (but far less sophisticated than) Linux.  The
simplicity of sv6 allowed us to quickly experiment with and iterate on
\sys's design, since some aspects of \sys's design require global
changes to the entire kernel.  Since sv6 is a monolithic kernel,
our prototype was able to tackle hard problems brought up by kernel
services such as a file system and a POSIX virtual memory system.

We evaluate the performance of our \sys prototype using
\bench~\cite{lebench}, which represents the most important system calls
for a range of application workloads: Spark, Redis, PostgreSQL,
Chromium, and building the Linux kernel.  \bench allows us to precisely
measure the impact of mitigations on system calls that matter
for applications.  The most recent Intel CPUs (such as Cascade Lake)
include hardware mitigations that cannot be fully disabled; however, some of
these mitigations are not needed in \sys.  To avoid the performance
overhead of such unnecessary mitigations, we run experiments on
the previous generation of Intel CPUs (Skylake).
%\autoref{s:motivation} discusses these hardware mitigation overheads in
%more detail.

\sys can run the \bench microbenchmarks with small performance
overheads compared to a kernel without mitigations.  For 18 out of
the 30 \bench microbenchmarks, \sys's performance is within 5\% of the
benchmark's performance without any mitigations (but at the cost of some
extra memory overhead).  In the worst case, the overhead is 4.3$\times$
(context switching between processes, where mitigations are unavoidable).
In contrast, standard mitigations incur a median overhead of 19\%, and
a worst case of nearly 7$\times$.  To confirm that \bench results translate
into application performance improvements, we measured the performance
of \texttt{git status}, which incurs 11.2\% overhead in \sys, compared
to 24.6\% with standard mitigations.

%We also report the results of
%a qualitative analysis of known transient execution vulnerabilities,
%which demonstrates that \sys's use of the unmapped speculation contract
%is able to handle all of them on recent x86 hardware and microcode.

%% The performance and security results for the \sys prototype suggest that
%% the \sys approach is a promising one for mitigating the performance overheads
%% of transient execution vulnerabilities.  \sys's design highlights a
%% number of important lessons in re-architecting the kernel to partition
%% the kernel memory across per-process kernel address spaces, which is
%% needed to take advantage of the unmapped speculation contract.  We hope
%% that these lessons will be valuable in re-architecting production kernels,
%% such as Linux, to achieve similar performance gains on real applications.

One of the limitations of \contract{} is that it does not cover all
possible transient execution attacks.  In particular, attacks where
the sensitive information is already present in the architectural or
microarchitectural state of the CPU are not covered by \contract{}.
For instance, the Spectre v3a attack can leak the sensitive contents of
a system register (MSR), instead of leaking sensitive data from memory.
\contract{} does not cover sensitive data that is stored outside of
memory, and \sys applies other mitigations (e.g., as in Linux) to address
those attacks.

%% Another limitation of our evaluation is that we cannot
%% selectively control hardware mitigations in the most recent Intel CPUs,
%% which forces us to use an older processor design for most experiments.
